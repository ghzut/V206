\section{Theorie}
\subsection{Die Wärmepumpe}
Dem 2. Hauptsatz der Thermodynamik nach fließt eine Wärmemenge nur von  einem  wärmeren in ein kälteres Wärmereservoir, solange keine mechanische Arbeit $W$ verrichtet wird.
Bei der Wärmepumpe wird nun mithilfe dieser Arbeit aus einem kälteren Reservoir die Wärmemenge $Q_.{1}$ entnommen und an das wärmere abgegeben. Dem 1. Hauptsatz der Thermodynamik zufolge gilt für diese vom wärmeren Reservoir aufgenommene Wärmemenge: \begin{equation}
Q_.{2} = Q_.{1} + W. \label{eq:q2}
\end{equation}
Das Verhältnis 
\begin{equation}
\nu = \frac{Q_.{2}}{W} \label{eq:ny1}
\end{equation}
bezeichnet die Güteziffer der Wärmepumpe.
Wenn der Prozess der Wärmeübertragung ohne Verluste, also umkehrbar, verläuft, gilt außerdem die Beziehung
\begin{equation}
\frac{Q_.{1}}{T_.{1}}-\frac{Q_.{2}}{T_.{2}} = 0 \cite{V206}
\end{equation}
Mit \eqref{eq:q2} folgt:
\begin{align*}
Q_.{2} &= \frac{T_.{2}}{T_.{1}}Q_.{2} + W
	   &= W \frac{T_.{1}}{T_.{1}-T_.{2}}
\end{align*}
und damit für die Güteziffer einer idealen Wärmepumpe
\begin{equation*}
\nu_.{ideal} = \frac{T_.{1}}{T_.{1}-T_.{2}} \cite{V206}
\end{equation*}
Da bei einer realen Wärmepumpe die Wärmeübertragung irreversibel verläuft, gilt in diesem Fall
\begin{equation}
\frac{Q_.{1}}{T_.{1}}-\frac{Q_.{2}}{T_.{2}} > 0 \label{eq:diffreal}
\end{equation}
und mit \eqref{eq:q2} und \eqref{eq:ny1}
\begin{equation}
\nu_.{real} < \frac{T_.{1}}{T_.{1}-T_.{2}} \label{eq:nyreal} \cite{V206}
\end{equation}
für ähnliche Temperaturen $T_.{1}$ und $T_.{2}$ ist dieser Prozess energetisch wesentlich günstiger, als die direkte Umwandlung von Arbeit in Wärme.
\subsubsection{Die reale Güteziffer}
Die pro Zeiteinheit gewonnene Wärmemenge lässt sich mit der Temperaturänderung $frac{\Delta T_.{1}}{\Delta t}$ im wärmeren Reservoir und aus den Wärmekapazitäten des in diesem enthaltenen Wassers und der Kupferschlange errechnen als
\[\frac{\Delta Q_.{2}}{\Delta t}=(m_.{1}c_.{W} +m_.{K}c_.{K})frac{\Delta T_.{1}}{\Delta t}\cite{V206}.\]
Daraus folgt für die reale Güteziffer
\begin{equation}
\nu = \frac{\Delta Q_.{2}}{\Delta t P}\label{eq:ny}
\end{equation}
$P$ ist dabei die gemittelte Leistungsaufnahme des Kompressors \cite{V206}
\subsubsection{Der Massendurchsatz}
Durch den Differenzenquotienten $frac{\Delta T_.{2}}{\Delta t}$
des kälteren Reservoirs und den Wärmekapazitäten lässt sich die pro Zeiteinheit entnommene Wärmemenge bestimmen als
\[\frac{\Delta Q_.{1}}{\Delta t}=(m_.{1}c_.{W} +m_.{K}c_.{K})frac{\Delta T_.{2}}{\Delta t}\cite{V206}.\]
Ist die Verdampfungswärme $L$ des zum Wärmetransport benutzten Mediums bekannt, kann der Massendurchsatz berechnet werden durch
\begin{equation*}
\frac{\Delta m}}{\Delta t} = \frac{\Delta Q_.{1}}{\Delta t L},\label{eq:Md1}
\end{equation*}
also
\begin{equation}
\frac{\Delta m}}{\Delta t} = (m_.{1}c_.{W} +m_.{K}c_.{K})frac{\Delta T_.{2}}{\Delta t L}.\label{eq:Md2}
\end{equation}
\subsubsection{Die mechanische Kompressorleistung}
Bei der Verringerung eines Volumens durch einen Kompressor gilt für die geleistete Arbeit:
\[W_.{mech}= - \int_{V_.{a}}^{V_.{b}} p\,\mathrm{d}V.\]
Ist diese Kompression adiabatisch, so folgt mit der Poisson-Gleichung
\[p_.{a}V_.{a}^{\frac{C_.{p}}{C_.{V}}} = p_.{b}V_.{b}^{\frac{C_.{p}}{C_.{V}}} = pV^{\frac{C_.{p}}{C_.{V}}} =pV^{\kappa}\]
und
\[P_.{mech} = \frac{\mathrm{d}A_.{mech}}{\mathrm{d}t}\]
\begin{equation}
P_.{mech} = \frac{1}{\kappa - 1}\left(p_.{b}sqrt[\kappa]{\frac{p_.{b}}{p_.{a}}} - p_.{a}\right)\frac{1}{\rho}\frac{\Delta m}}{\Delta t},\label{eq:P}
\end{equation}
wobei $\rho$ die Gasdichte des Transportmediums bei Druck $p_.{a}$ ist.